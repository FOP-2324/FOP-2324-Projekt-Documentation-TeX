\documentclass[
    inlineshortcut=java, % Befehl \inlinejava{<code>} Konfigurieren
    corporatedesign, % TU-Design
    boxarc, % Abgerundete Ecken bei den Boxen
    fop, % FOP-Vorlage benutzen
    % dark_mode, % don't comment this in, use environment variable instead. If commented in, the makefile will not work
]{algoexercise}

%%------------%%
%%--Packages--%%
%%------------%%

\usepackage{enumitem}
\usepackage{rotating}
\usepackage{hyperref}
\usepackage{placeins}
\setlength{\marginparwidth}{10cm}
\usetikzlibrary{positioning}

%%----------------------------%%
%%--Stilistische Anpassungen--%%
%%----------------------------%%

\ConfigureHeadline{
    headline={algo-min}
}

\DeclareDocumentProperty{duedate}[][Abgabe~des~Projekts:\hfill *]

\defcaptionname{ngerman, german}{\PointName}{Projektpunkt}
\defcaptionname{english, american, british}{\PointName}{Projectpoint}
\defcaptionname{ngerman, german}{\PointsName}{Projektpunkte}
\defcaptionname{english, american, british}{\PointsName}{Projectpoints}

\ExplSyntaxOn
\renewcommand*{\author}[1]{
    \exp_args:Nc \seq_gset_split:Nnn {g_ \tudaPkgPrefix _author_seq} {\and} {#1}
    \seq_if_empty:cF {g_ \tudaPkgPrefix _author_seq} {\tl_gset:Nn \printAuthor {\int_compare:nTF{\seq_count:c {g_ \tudaPkgPrefix _author_seq} > 1}{Gruppenmitglieder}{Gruppenmitglied}:~\hfill\seq_use:cnnn {g_ \tudaPkgPrefix _author_seq} {~\authorandname{}~} {,~} {~\authorandname{}~}\par}}
}
\ExplSyntaxOff

\def\tileWidth{10mm}
\pgfmathsetlengthmacro\tileRadius{\tileWidth/cos(30)}
\def\gridRadius{3}
\colorlet{tilecolor}{\IfDarkModeTF{gray!60}{gray}}
\pgfmathdeclarerandomlist{ressourceTypes}{
    {WOOD}
    {CLAY}
    {WOOL}
    {GRAIN}
    {ORE}
}

\tikzset{
fbw/.style={very thick,shorten >=-0.6pt,shorten <=-0.6pt},
fbwshorten/.style={fbw,shorten >=0.6pt,shorten <=0.6pt},
tile/.style={
    tilecolor,
    regular polygon,
    regular polygon sides=6,
    shape border rotate=30,
    minimum size=2*\tileRadius,
    inner sep=0mm,
    outer sep=0mm,
    draw=tilecolor,
    line width=1mm
},
road/.style={
    draw=#1,
    line width=0.5mm,
    shorten >=1pt,shorten <=1pt,
    % densely dashed,
},
road/.default=.,
settlement/.style={
    draw=tilecolor,
    fill=\thepagecolor,
    circle,
    minimum size=4mm,
    inner sep=0mm,
    outer sep=0mm,
    line width=.5mm,
},
settlement/.default=.,
village/.style={
settlement=#1,
label=center:{\textcolor{#1}{\tiny\faHome}},
},
village/.default=.,
city/.style={
settlement=#1,
label=center:{\textcolor{#1}{\tiny\faCity}},
},
city/.default=.,
}
\colorlet{qaxiscolor}{TUDa-3\IfDarkModeTF{a}{b}}
\colorlet{raxiscolor}{TUDa-1\IfDarkModeTF{a}{b}}
\colorlet{saxiscolor}{TUDa-10\IfDarkModeTF{a}{b}}

\newcommand*{\qaxis}{\ensuremath{{\color{qaxiscolor}q}}}
\newcommand*{\raxis}{\ensuremath{{\color{raxiscolor}r}}}
\newcommand*{\saxis}{\ensuremath{{\color{saxiscolor}s}}}
\newcommand*{\hexcoord}[3]{({\color{qaxiscolor}#1}, {\color{raxiscolor}#2}, {\color{saxiscolor}#3})}

\pgfmathsetlengthmacro\xdiff{\tileWidth/2}
\pgfmathsetlengthmacro\ydiff{sin(60) * \xdiff}
\pgfmathsetlengthmacro\tileRadius{\xdiff/cos(30)}

\newcommand*{\roads}[2]{
    % define boolean for first iteration
    \def\lastpos{}
    \foreach \r in {#2}{
        % if \lastpos is not empty
        \ifx\lastpos\empty
        \else
            % draw road
            \draw[road=#1] \lastpos -- \r;
        \fi
        % set last position
        \xdef\lastpos{\r}
    }
}

%%---------------------------%%
%%--Dokumenteneinstellungen--%%
%%---------------------------%%

\def\groupnumber{<GroupNumber>} % Gruppennummer
\author{<Author1>\and <Author2>\and <Author3>} % Gruppenmitglieder
\duedate{15.03.2024 bis 23:50 Uhr}
\subtitle{Prof. Karsten Weihe}
\dozent{Prof. Karsten Weihe} % chktex 12
\fachbereich{Informatik}
\semester{Wintersemester 23/24} % z.B. SoSe 2022 oder WiSe 2022/2023
\sheetnumber*{Projekt} % Einstellige Nummern werden mit 0 aufgefüllt
\slides{*} % Die Relevanten Foliensätze
\topics{Alle Inhalte der FOP} % Für das Übungsblatt relevante Themengebiete
\title[Gruppe \groupnumber{}]{Projekt-Dokumentation}
\version{1.0.0-SNAPSHOT}
\graphicspath{{./pictures/}}

%%-------------------------%%
%%--Beginn des Dokumentes--%%
%%-------------------------%%

\begin{document}%

    %%-----------%%
    %%--Titelei--%%
    %%-----------%%

    \maketitle{}

    %%-------------%%
    %%--H-Übungen--%%
    %%-------------%%

    \hue{FOP \getSheetnumber{}}{Die Siedler von Catan}{100}

    %\tableofcontents

    %%--------------%%
    %%--Einleitung--%%
    %%--------------%%

    Dies ist die \LaTeX-Vorlage für das FOP-Projekt. Alle notwendigen Einstellungen können sie im Bereich Dokumenteneinstellungen ab Z. 126 anpassen:

    \begin{codeBlock}[escapeinside=||,highlightlines=1-2]{title=\codeBlockTitle{Dokumenteneinstellungen},minted language=latex}
        \def\groupnumber{<GroupNumber>} % Gruppennummer
        \author{<Author1>\and <Author2>\and <Author3>} % Gruppenmitglieder
        \duedate{15.03.2024 bis 23:50 Uhr}
        \subtitle{Prof. Karsten Weihe}
        \dozent{Prof. Karsten Weihe} % chktex 12
        \fachbereich{Informatik}
        \semester{Wintersemester 23/24} % z.B. SoSe 2022 oder WiSe 2022/2023
        \sheetnumber*{Projekt} % Einstellige Nummern werden mit 0 aufgefüllt
        \slides{*} % Die Relevanten Foliensätze
        \topics{Alle Inhalte der FOP} % Für das Übungsblatt relevante Themengebiete
        \title[Gruppe \groupnumber{}]{Projekt-Dokumentation}
        \version{1.0.0-SNAPSHOT}
        \graphicspath{{./pictures/}}
    \end{codeBlock}

    Falls Sie unsere \LaTeX Zeichnungen verwenden oder erweitern wollen, sind hier einige Beispiele:
    \begin{figure}[ht!]
        \centering
        \begin{tikzpicture}
            \tikzset{
                x={(\xdiff, \ydiff)},
                y={(0, -\ydiff)},
                z={(-\xdiff, \ydiff)},
            }
            % draw the hexagon grid with given radius
            \foreach \x in {-\gridRadius, ..., \gridRadius}{
                \foreach \y in {-\gridRadius, ..., \gridRadius}{
                    % only draw if in radius
                    \pgfmathtruncatemacro{\sum}{abs(\x+\y)}
                    \ifnum\sum < \numexpr\gridRadius+1\relax
                        \def\z{\the\numexpr-\x-\y\relax}
                        % select random tile type
                        \pgfmathrandomitem{\type}{ressourceTypes}
                        % tile
                        \node[tile,font=\tiny,align=center] (g-\x-\y) at (\x, \y, \z){(\x,\y,\z)\\\type};
                    \fi
                }
            }

            % roads
            \roads{cyan}{
                (g-0-0.corner 1),
                (g-0-0.corner 2),
                (g-0-0.corner 3),
                (g--1-0.corner 4)
            }

            \roads{orange}{
                (g-1-2.corner 2),
                (g-1-2.corner 3),
                (g-0-2.corner 4)
            }

            \roads{orange}{
                (g-0-0.corner 3),
                (g-0-0.corner 4)
            }

            % villages and cities
            \node[village=cyan] at (g-0-0.corner 1){};
            \node[city=cyan] at (g--1-0.corner 4){};
            \node[village=orange] at (g-0-0.corner 4){};
            \node[city=orange] at (g-1-2.corner 2){};

            % compass
            \coordinate (compass) at (-5,3.5,3);
            \draw[qaxiscolor,thick,-Latex] (compass)+(-1,0,0) -- ++(1.2,0,0) node[above right]{\verb+q+};
            \draw[raxiscolor,thick,-Latex] (compass)+(0,-1,0) -- ++(0,1.5,0) node[below]{\verb+r+};
            \draw[saxiscolor,thick,-Latex] (compass)+(0,0,-1) -- ++(0,0,1.2) node[above left]{\verb+s+};

        \end{tikzpicture}
        \label{fig:hexgrid}
        \caption[short]{Die Abbildung zeigt eine vereinfachte Darstellung des Spielfelds. Koordinaten werden im Format \hexcoord{q}{r}{s} angegeben. Hierbei steht \inlinejava{WOOD} für Holz, \inlinejava{CLAY} für Lehm, \inlinejava{GRAIN} für Getreide, ORE für Erz und \inlinejava{WOOL} für Wolle.}
    \end{figure}

    \begin{figure}[h]
        \centering
        \begin{subfigure}[b]{\textwidth/3}
            \centering
            \begin{tikzpicture}
                \tikzset{
                    x={(\xdiff, \ydiff)},
                    y={(0, -\ydiff)},
                    z={(-\xdiff, \ydiff)},
                }
                % draw the hexagon grid with given radius
                \pgfmathrandomitem{\type}{ressourceTypes}
                \node[tile,font=\tiny,align=center] (g-0-0) at (0, 0, 0){(0,0,0)\\\type};
                \node[tile,font=\tiny,align=center] (g-0--1) at (0, -1, 1){(0,-1,1)\\\type};
                \node[tile,font=\tiny,align=center] (g-1--1) at (1, -1, 0){(1,-1,0)\\\type};

                \node[TUDa-8a,fill,circle, minimum size=2mm, inner sep=0pt, outer sep=0pt] at (g-0-0.corner 1) (mark-1){};
            \end{tikzpicture}
            \caption{ohne Siedlung (orange markiert)}
            \label{fig:intersection:withoutSettlement}
        \end{subfigure}%
        \begin{subfigure}[b]{\textwidth/3}
            \centering
            \begin{tikzpicture}
                \tikzset{
                    x={(\xdiff, \ydiff)},
                    y={(0, -\ydiff)},
                    z={(-\xdiff, \ydiff)},
                }
                % draw the hexagon grid with given radius
                \pgfmathrandomitem{\type}{ressourceTypes}
                \node[tile,font=\tiny,align=center] (g-0-0) at (0, 0, 0){(0,0,0)\\\type};
                \node[tile,font=\tiny,align=center] (g-0--1) at (0, -1, 1){(0,-1,1)\\\type};
                \node[tile,font=\tiny,align=center] (g-1--1) at (1, -1, 0){(1,-1,0)\\\type};

                \node[village=TUDa-8a] at (g-0-0.corner 1){};
            \end{tikzpicture}
            \caption{mit Dorf}
            \label{fig:intersection:withVillage}
        \end{subfigure}%
        \begin{subfigure}[b]{\textwidth/3}
            \centering
            \begin{tikzpicture}
                \tikzset{
                    x={(\xdiff, \ydiff)},
                    y={(0, -\ydiff)},
                    z={(-\xdiff, \ydiff)},
                }
                % draw the hexagon grid with given radius
                \pgfmathrandomitem{\type}{ressourceTypes}
                \node[tile,font=\tiny,align=center] (g-0-0) at (0, 0, 0){(0,0,0)\\\type};
                \node[tile,font=\tiny,align=center] (g-0--1) at (0, -1, 1){(0,-1,1)\\\type};
                \node[tile,font=\tiny,align=center] (g-1--1) at (1, -1, 0){(1,-1,0)\\\type};

                \node[city=TUDa-8a] at (g-0-0.corner 1){};
            \end{tikzpicture}
            \caption{mit Stadt}
            \label{fig:intersection:withCity}
        \end{subfigure}%
        \caption[short]{Die Abbildung zeigt eine Intersection und die drei angrenzenden Tiles. Eine Intersection kann auch weniger als drei angrenzende Tiles haben, wenn sie sich am Spielfeldrand befindet. Um genau zu sein: An jeder Ecke jedes \inlinejava{Tile}s befindet sich eine Intersection.}
        \label{fig:intersection}
    \end{figure}
    \FloatBarrier{}

    \begin{vanforderung}
        Entfernen Sie die Beispiele vor der Abgabe.
    \end{vanforderung}

    \begin{hinweis}
        Auskommentieren reicht
    \end{hinweis}

    \clearpage{}

    % direkt mit H4 beginnen
    \setcounter{task}{3}
    \begin{task}[points=auto]{Weiterführende Aufgaben}\label{ex:H4} % Points=auto won't work with old version of tuda-ci, if it doesn't work for you, set it to 30

        \begin{subtask}[title=Weiterentwicklung des GUIs]\label{ex:H4.1}
            TODO
        \end{subtask}

        \begin{subtask}[title=Neuer Rohstoff,points=5]\label{ex:H4.3}
            TODO
        \end{subtask}

        \begin{subtask}[title=Neue Gebäudestruktur,points=5]\label{ex:H4.4}
            TODO
        \end{subtask}

        \begin{subtask}[title={Neue Entwicklungskarte},points=5]\label{ex:H4.2}
            TODO
        \end{subtask}

        \begin{subtask}[title=Neue Spielmechanik,points=5]\label{ex:H4.5}
            TODO
        \end{subtask}

        \begin{subtask}[title=AI als Gegner?,points=5]\label{ex:H4.6}
            TODO
        \end{subtask}
    \end{task}
\end{document}
